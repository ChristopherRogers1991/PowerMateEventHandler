%
% API Documentation for API Documentation
% Module PowerMateEventHandler
%
% Generated by epydoc 3.0.1
% [Sun Mar  8 20:39:42 2015]
%

%%%%%%%%%%%%%%%%%%%%%%%%%%%%%%%%%%%%%%%%%%%%%%%%%%%%%%%%%%%%%%%%%%%%%%%%%%%
%%                          Module Description                           %%
%%%%%%%%%%%%%%%%%%%%%%%%%%%%%%%%%%%%%%%%%%%%%%%%%%%%%%%%%%%%%%%%%%%%%%%%%%%

    \index{PowerMateEventHandler \textit{(module)}|(}
\section{Module PowerMateEventHandler}

    \label{PowerMateEventHandler}

%%%%%%%%%%%%%%%%%%%%%%%%%%%%%%%%%%%%%%%%%%%%%%%%%%%%%%%%%%%%%%%%%%%%%%%%%%%
%%                               Functions                               %%
%%%%%%%%%%%%%%%%%%%%%%%%%%%%%%%%%%%%%%%%%%%%%%%%%%%%%%%%%%%%%%%%%%%%%%%%%%%

  \subsection{Functions}

    \label{PowerMateEventHandler:find_device}
    \index{PowerMateEventHandler \textit{(module)}!PowerMateEventHandler.find\_device \textit{(function)}}

    \vspace{0.5ex}

\hspace{.8\funcindent}\begin{boxedminipage}{\funcwidth}

    \raggedright \textbf{find\_device}(\textit{dev\_dir}={\tt \texttt{'}\texttt{/dev/input/}\texttt{'}})

    \vspace{-1.5ex}

    \rule{\textwidth}{0.5\fboxrule}
\setlength{\parskip}{2ex}
    Finds and returns the device in dev\_dir

    If the user does not have permission to access a device in dev\_dir, an
    OSError Exception will be raised.

    OSErrors are printed to stderr. These will likely happen if the user 
    does not have permission to all devices. If the function retrns None 
    with the device plugged in, check the permissions on the device. 
    (There's probably a better way to do this - check the devices before 
    attempting to open them - but that will have to wait for the moment.)

\setlength{\parskip}{1ex}
      \textbf{Return Value}
    \vspace{-1ex}

      \begin{quote}
      An evdev.InputDevice. None if the device is not found.

      {\it (type=evdev.InputDevice or None)}

      \end{quote}

    \end{boxedminipage}

    \label{PowerMateEventHandler:event_time_in_ms}
    \index{PowerMateEventHandler \textit{(module)}!PowerMateEventHandler.event\_time\_in\_ms \textit{(function)}}

    \vspace{0.5ex}

\hspace{.8\funcindent}\begin{boxedminipage}{\funcwidth}

    \raggedright \textbf{event\_time\_in\_ms}(\textit{event})

    \vspace{-1.5ex}

    \rule{\textwidth}{0.5\fboxrule}
\setlength{\parskip}{2ex}
\setlength{\parskip}{1ex}
      \textbf{Parameters}
      \vspace{-1ex}

      \begin{quote}
        \begin{Ventry}{xxxxx}

          \item[event]

          the event to get the time for

            {\it (type=evdev.InputEvent)}

        \end{Ventry}

      \end{quote}

      \textbf{Return Value}
    \vspace{-1ex}

      \begin{quote}
      The time in ms the event occurred (as an int)

      {\it (type=int)}

      \end{quote}

\textbf{Note:} Does this by converting the event microseconds (event.usec) to seconds 
(multiply by 1000000), adding the event seconds (event.sec), converting to 
ms (multiply by 1000), then casting to an int.



    \end{boxedminipage}

    \label{PowerMateEventHandler:get_uinput}
    \index{PowerMateEventHandler \textit{(module)}!PowerMateEventHandler.get\_uinput \textit{(function)}}

    \vspace{0.5ex}

\hspace{.8\funcindent}\begin{boxedminipage}{\funcwidth}

    \raggedright \textbf{get\_uinput}(\textit{dev})

    \vspace{-1.5ex}

    \rule{\textwidth}{0.5\fboxrule}
\setlength{\parskip}{2ex}
\setlength{\parskip}{1ex}
      \textbf{Parameters}
      \vspace{-1ex}

      \begin{quote}
        \begin{Ventry}{xxx}

          \item[dev]

          An evdev.InputDevice for the PowerMate (see find\_device)

            {\it (type=evdev.InputDevice:)}

        \end{Ventry}

      \end{quote}

      \textbf{Return Value}
    \vspace{-1ex}

      \begin{quote}
      An evdev.UInput for the device. This can be used to write to the 
      device (to change the led brightness).

      {\it (type=evdev.UInput)}

      \end{quote}

    \end{boxedminipage}


%%%%%%%%%%%%%%%%%%%%%%%%%%%%%%%%%%%%%%%%%%%%%%%%%%%%%%%%%%%%%%%%%%%%%%%%%%%
%%                               Variables                               %%
%%%%%%%%%%%%%%%%%%%%%%%%%%%%%%%%%%%%%%%%%%%%%%%%%%%%%%%%%%%%%%%%%%%%%%%%%%%

  \subsection{Variables}

    \vspace{-1cm}
\hspace{\varindent}\begin{longtable}{|p{\varnamewidth}|p{\vardescrwidth}|l}
\cline{1-2}
\cline{1-2} \centering \textbf{Name} & \centering \textbf{Description}& \\
\cline{1-2}
\endhead\cline{1-2}\multicolumn{3}{r}{\small\textit{continued on next page}}\\\endfoot\cline{1-2}
\endlastfoot\raggedright B\-U\-T\-T\-O\-N\-\_\-P\-U\-S\-H\-E\-D\- & \raggedright \textbf{Value:} 
{\tt 256}&\\
\cline{1-2}
\raggedright K\-N\-O\-B\-\_\-T\-U\-R\-N\-E\-D\- & \raggedright \textbf{Value:} 
{\tt 7}&\\
\cline{1-2}
\raggedright P\-O\-S\-I\-T\-I\-V\-E\- & \raggedright \textbf{Value:} 
{\tt 1}&\\
\cline{1-2}
\raggedright N\-E\-G\-A\-T\-I\-V\-E\- & \raggedright \textbf{Value:} 
{\tt -1}&\\
\cline{1-2}
\raggedright t\-i\-m\-e\-\_\-d\-o\-w\-n\- & \raggedright \textbf{Value:} 
{\tt 0}&\\
\cline{1-2}
\raggedright l\-e\-d\-\_\-b\-r\-i\-g\-h\-t\-n\-e\-s\-s\- & \raggedright \textbf{Value:} 
{\tt 100}&\\
\cline{1-2}
\raggedright f\-l\-a\-s\-h\-\_\-d\-u\-r\-a\-t\-i\-o\-n\- & \raggedright \textbf{Value:} 
{\tt 0.15}&\\
\cline{1-2}
\raggedright \_\-\_\-p\-a\-c\-k\-a\-g\-e\-\_\-\_\- & \raggedright \textbf{Value:} 
{\tt None}&\\
\cline{1-2}
\end{longtable}


%%%%%%%%%%%%%%%%%%%%%%%%%%%%%%%%%%%%%%%%%%%%%%%%%%%%%%%%%%%%%%%%%%%%%%%%%%%
%%                           Class Description                           %%
%%%%%%%%%%%%%%%%%%%%%%%%%%%%%%%%%%%%%%%%%%%%%%%%%%%%%%%%%%%%%%%%%%%%%%%%%%%

    \index{PowerMateEventHandler \textit{(module)}!PowerMateEventHandler.ConsolidatedEventCode \textit{(class)}|(}
\subsection{Class ConsolidatedEventCode}

    \label{PowerMateEventHandler:ConsolidatedEventCode}
\begin{tabular}{cccccccc}
% Line for object, linespec=[False, False]
\multicolumn{2}{r}{\settowidth{\BCL}{object}\multirow{2}{\BCL}{object}}
&&
&&
  \\\cline{3-3}
  &&\multicolumn{1}{c|}{}
&&
&&
  \\
% Line for enum.Enum, linespec=[False]
\multicolumn{4}{r}{\settowidth{\BCL}{enum.Enum}\multirow{2}{\BCL}{enum.Enum}}
&&
  \\\cline{5-5}
  &&&&\multicolumn{1}{c|}{}
&&
  \\
&&&&\multicolumn{2}{l}{\textbf{PowerMateEventHandler.ConsolidatedEventCode}}
\end{tabular}

SINGLE\_CLICK = 0 DOUBLE\_CLICK = SINGLE\_CLICK + 1 LONG\_CLICK = 
DOUBLE\_CLICK + 1 RIGHT\_TURN = LONG\_CLICK + 1 LEFT\_TURN = RIGHT\_TURN + 
1


%%%%%%%%%%%%%%%%%%%%%%%%%%%%%%%%%%%%%%%%%%%%%%%%%%%%%%%%%%%%%%%%%%%%%%%%%%%
%%                                Methods                                %%
%%%%%%%%%%%%%%%%%%%%%%%%%%%%%%%%%%%%%%%%%%%%%%%%%%%%%%%%%%%%%%%%%%%%%%%%%%%

  \subsubsection{Methods}

    \vspace{0.5ex}

\hspace{.8\funcindent}\begin{boxedminipage}{\funcwidth}

    \raggedright \textbf{\_\_format\_\_}(\textit{self}, \textit{format\_spec})

\setlength{\parskip}{2ex}
    default object formatter

\setlength{\parskip}{1ex}
      Overrides: object.\_\_format\_\_ 	extit{(inherited documentation)}

    \end{boxedminipage}

    \vspace{0.5ex}

\hspace{.8\funcindent}\begin{boxedminipage}{\funcwidth}

    \raggedright \textbf{\_\_reduce\_ex\_\_}(\textit{self}, \textit{proto})

\setlength{\parskip}{2ex}
    helper for pickle

\setlength{\parskip}{1ex}
      Overrides: object.\_\_reduce\_ex\_\_ 	extit{(inherited documentation)}

    \end{boxedminipage}

    \vspace{0.5ex}

\hspace{.8\funcindent}\begin{boxedminipage}{\funcwidth}

    \raggedright \textbf{\_\_repr\_\_}(\textit{self})

\setlength{\parskip}{2ex}
    repr(x)

\setlength{\parskip}{1ex}
      Overrides: object.\_\_repr\_\_ 	extit{(inherited documentation)}

    \end{boxedminipage}

    \vspace{0.5ex}

\hspace{.8\funcindent}\begin{boxedminipage}{\funcwidth}

    \raggedright \textbf{\_\_str\_\_}(\textit{self})

\setlength{\parskip}{2ex}
    str(x)

\setlength{\parskip}{1ex}
      Overrides: object.\_\_str\_\_ 	extit{(inherited documentation)}

    \end{boxedminipage}


\large{\textbf{\textit{Inherited from enum.Enum}}}

\begin{quote}
\_\_dir\_\_(), \_\_eq\_\_(), \_\_ge\_\_(), \_\_gt\_\_(), \_\_hash\_\_(), \_\_le\_\_(), \_\_lt\_\_(), \_\_ne\_\_(), \_\_new\_\_()
\end{quote}

\large{\textbf{\textit{Inherited from object\textit{(Section \ref{object})}}}}

\begin{quote}
\_\_delattr\_\_(), \_\_getattribute\_\_(), \_\_init\_\_(), \_\_reduce\_\_(), \_\_setattr\_\_(), \_\_sizeof\_\_(), \_\_subclasshook\_\_()
\end{quote}

%%%%%%%%%%%%%%%%%%%%%%%%%%%%%%%%%%%%%%%%%%%%%%%%%%%%%%%%%%%%%%%%%%%%%%%%%%%
%%                              Properties                               %%
%%%%%%%%%%%%%%%%%%%%%%%%%%%%%%%%%%%%%%%%%%%%%%%%%%%%%%%%%%%%%%%%%%%%%%%%%%%

  \subsubsection{Properties}

    \vspace{-1cm}
\hspace{\varindent}\begin{longtable}{|p{\varnamewidth}|p{\vardescrwidth}|l}
\cline{1-2}
\cline{1-2} \centering \textbf{Name} & \centering \textbf{Description}& \\
\cline{1-2}
\endhead\cline{1-2}\multicolumn{3}{r}{\small\textit{continued on next page}}\\\endfoot\cline{1-2}
\endlastfoot\multicolumn{2}{|l|}{\textit{Inherited from object \textit{(Section \ref{object})}}}\\
\multicolumn{2}{|p{\varwidth}|}{\raggedright \_\_class\_\_}\\
\cline{1-2}
\end{longtable}


%%%%%%%%%%%%%%%%%%%%%%%%%%%%%%%%%%%%%%%%%%%%%%%%%%%%%%%%%%%%%%%%%%%%%%%%%%%
%%                            Class Variables                            %%
%%%%%%%%%%%%%%%%%%%%%%%%%%%%%%%%%%%%%%%%%%%%%%%%%%%%%%%%%%%%%%%%%%%%%%%%%%%

  \subsubsection{Class Variables}

    \vspace{-1cm}
\hspace{\varindent}\begin{longtable}{|p{\varnamewidth}|p{\vardescrwidth}|l}
\cline{1-2}
\cline{1-2} \centering \textbf{Name} & \centering \textbf{Description}& \\
\cline{1-2}
\endhead\cline{1-2}\multicolumn{3}{r}{\small\textit{continued on next page}}\\\endfoot\cline{1-2}
\endlastfoot\raggedright S\-I\-N\-G\-L\-E\-\_\-C\-L\-I\-C\-K\- & \raggedright \textbf{Value:} 
{\tt 0}&\\
\cline{1-2}
\raggedright D\-O\-U\-B\-L\-E\-\_\-C\-L\-I\-C\-K\- & \raggedright \textbf{Value:} 
{\tt SINGLE\_CLICK+ 1}&\\
\cline{1-2}
\raggedright L\-O\-N\-G\-\_\-C\-L\-I\-C\-K\- & \raggedright \textbf{Value:} 
{\tt DOUBLE\_CLICK+ 1}&\\
\cline{1-2}
\raggedright R\-I\-G\-H\-T\-\_\-T\-U\-R\-N\- & \raggedright \textbf{Value:} 
{\tt LONG\_CLICK+ 1}&\\
\cline{1-2}
\raggedright L\-E\-F\-T\-\_\-T\-U\-R\-N\- & \raggedright \textbf{Value:} 
{\tt RIGHT\_TURN+ 1}&\\
\cline{1-2}
\raggedright \_\-m\-e\-m\-b\-e\-r\-\_\-m\-a\-p\-\_\- & \raggedright \textbf{Value:} 
{\tt OrderedDict([('SINGLE\_CLICK', {\textless}ConsolidatedEventCode.SING\texttt{...}}&\\
\cline{1-2}
\raggedright \_\-m\-e\-m\-b\-e\-r\-\_\-n\-a\-m\-e\-s\-\_\- & \raggedright \textbf{Value:} 
{\tt \texttt{[}\texttt{'}\texttt{SINGLE\_CLICK}\texttt{'}\texttt{, }\texttt{'}\texttt{DOUBLE\_CLICK}\texttt{'}\texttt{, }\texttt{'}\texttt{LONG\_CLICK}\texttt{'}\texttt{, }\texttt{'}\texttt{RIGHT\_TUR}\texttt{...}}&\\
\cline{1-2}
\raggedright \_\-v\-a\-l\-u\-e\-2\-m\-e\-m\-b\-e\-r\-\_\-m\-a\-p\-\_\- & \raggedright \textbf{Value:} 
{\tt \texttt{\{}0\texttt{: }{\textless}ConsolidatedEventCode.SINGLE\_CLICK: 0{\textgreater}\texttt{, }1\texttt{: }{\textless}Consolid\texttt{...}}&\\
\cline{1-2}
\multicolumn{2}{|l|}{\textit{Inherited from enum.Enum}}\\
\multicolumn{2}{|p{\varwidth}|}{\raggedright name, value}\\
\cline{1-2}
\end{longtable}

    \index{PowerMateEventHandler \textit{(module)}!PowerMateEventHandler.ConsolidatedEventCode \textit{(class)}|)}

%%%%%%%%%%%%%%%%%%%%%%%%%%%%%%%%%%%%%%%%%%%%%%%%%%%%%%%%%%%%%%%%%%%%%%%%%%%
%%                           Class Description                           %%
%%%%%%%%%%%%%%%%%%%%%%%%%%%%%%%%%%%%%%%%%%%%%%%%%%%%%%%%%%%%%%%%%%%%%%%%%%%

    \index{PowerMateEventHandler \textit{(module)}!PowerMateEventHandler.PowerMateEventHandler \textit{(class)}|(}
\subsection{Class PowerMateEventHandler}

    \label{PowerMateEventHandler:PowerMateEventHandler}

%%%%%%%%%%%%%%%%%%%%%%%%%%%%%%%%%%%%%%%%%%%%%%%%%%%%%%%%%%%%%%%%%%%%%%%%%%%
%%                                Methods                                %%
%%%%%%%%%%%%%%%%%%%%%%%%%%%%%%%%%%%%%%%%%%%%%%%%%%%%%%%%%%%%%%%%%%%%%%%%%%%

  \subsubsection{Methods}

    \label{PowerMateEventHandler:PowerMateEventHandler:__init__}
    \index{PowerMateEventHandler \textit{(module)}!PowerMateEventHandler.PowerMateEventHandler \textit{(class)}!PowerMateEventHandler.PowerMateEventHandler.\_\_init\_\_ \textit{(method)}}

    \vspace{0.5ex}

\hspace{.8\funcindent}\begin{boxedminipage}{\funcwidth}

    \raggedright \textbf{\_\_init\_\_}(\textit{self}, \textit{brightness}={\tt 255}, \textit{read\_delay}={\tt None}, \textit{turn\_delay}={\tt 0}, \textit{long\_press\_time}={\tt 0.5}, \textit{double\_click\_time}={\tt 0.3}, \textit{dev\_dir}={\tt \texttt{'}\texttt{/dev/input/}\texttt{'}})

    \vspace{-1.5ex}

    \rule{\textwidth}{0.5\fboxrule}
\setlength{\parskip}{2ex}
    Find the PowerMateDevice, and get set up to start reading from it and 
    writing to it.

    If the device is not found (can happen if the device is not plugged in,
    or the user does not have permissions to it) a DeviceNotFound Exception
    will be raised.

\setlength{\parskip}{1ex}
      \textbf{Parameters}
      \vspace{-1ex}

      \begin{quote}
        \begin{Ventry}{xxxxxxxxxxxxxxxxx}

          \item[brightness]

          The inital brightness of the led in the base.

            {\it (type=int)}

          \item[read\_delay]

          Timeout when waiting for the device to be readable. Having a time
          out allows the threads to be joinable without waiting for another
          event. None (default) means to wait indefinitely for the device 
          to be readable. This will probably yield the best performance, 
          but means the thread will not stop after a call to stop() until a
          new event is triggered.

          Having this configurable  was intendted to allow the reading of 
          events to be stoppable (i.e to keep from blocking the thread 
          indefinitely). It was made tunable to allow good performance on 
          fast CPUs, but not hog resources on slower machines.

          Setting delay to None will cause the thread to block 
          indefinitely.

            {\it (type=double)}

          \item[turn\_delay]

          Time in ms between consolidated turns.

            {\it (type=double)}

          \item[long\_press\_time]

          time (in s) the button must be held to register a long press

            {\it (type=double)}

          \item[double\_click\_time]

          time (in s) the button must be pressed again after a single press
          to register as a double

            {\it (type=double)}

          \item[dev\_dir]

          The directory in which to look for the device.

            {\it (type=str)}

        \end{Ventry}

      \end{quote}

    \end{boxedminipage}

    \label{PowerMateEventHandler:PowerMateEventHandler:set_led_brightness}
    \index{PowerMateEventHandler \textit{(module)}!PowerMateEventHandler.PowerMateEventHandler \textit{(class)}!PowerMateEventHandler.PowerMateEventHandler.set\_led\_brightness \textit{(method)}}

    \vspace{0.5ex}

\hspace{.8\funcindent}\begin{boxedminipage}{\funcwidth}

    \raggedright \textbf{set\_led\_brightness}(\textit{self}, \textit{brightness})

    \vspace{-1.5ex}

    \rule{\textwidth}{0.5\fboxrule}
\setlength{\parskip}{2ex}
    Sets the led in the base to the specified brightness. The valid range 
    is 0-255, where 0 is off. Anything less than 0 will be treated as zero,
    anything greater than 255 will be treated as 255.

\setlength{\parskip}{1ex}
    \end{boxedminipage}

    \label{PowerMateEventHandler:PowerMateEventHandler:flash_led}
    \index{PowerMateEventHandler \textit{(module)}!PowerMateEventHandler.PowerMateEventHandler \textit{(class)}!PowerMateEventHandler.PowerMateEventHandler.flash\_led \textit{(method)}}

    \vspace{0.5ex}

\hspace{.8\funcindent}\begin{boxedminipage}{\funcwidth}

    \raggedright \textbf{flash\_led}(\textit{self}, \textit{num\_flashes}={\tt 2}, \textit{brightness}={\tt 100}, \textit{duration}={\tt 0.15}, \textit{sleep}={\tt 0.15})

    \vspace{-1.5ex}

    \rule{\textwidth}{0.5\fboxrule}
\setlength{\parskip}{2ex}
    Convenience function to flash the led in the base. After the flashes, 
    the brightness will be reset to whatever it was when this function was 
    called.

\setlength{\parskip}{1ex}
      \textbf{Parameters}
      \vspace{-1ex}

      \begin{quote}
        \begin{Ventry}{xxxxxxxxxxx}

          \item[num\_flashes]

          number times to flash

            {\it (type=int)}

          \item[brightness]

          the brightness of the flashes (range defined by 
          set\_led\_brightness)

            {\it (type=int)}

          \item[duration]

          length of each flash in seconds (decimals accepted)

            {\it (type=double)}

          \item[sleep]

          time between each flash in seconds (decimals accepted)

            {\it (type=double)}

        \end{Ventry}

      \end{quote}

    \end{boxedminipage}

    \label{PowerMateEventHandler:PowerMateEventHandler:start}
    \index{PowerMateEventHandler \textit{(module)}!PowerMateEventHandler.PowerMateEventHandler \textit{(class)}!PowerMateEventHandler.PowerMateEventHandler.start \textit{(method)}}

    \vspace{0.5ex}

\hspace{.8\funcindent}\begin{boxedminipage}{\funcwidth}

    \raggedright \textbf{start}(\textit{self}, \textit{raw\_only}={\tt False})

    \vspace{-1.5ex}

    \rule{\textwidth}{0.5\fboxrule}
\setlength{\parskip}{2ex}
    Begin capturing/queueing events. Once this has been run, get\_next() 
    can be used to start pulling events off the queue.

\setlength{\parskip}{1ex}
    \end{boxedminipage}

    \label{PowerMateEventHandler:PowerMateEventHandler:stop}
    \index{PowerMateEventHandler \textit{(module)}!PowerMateEventHandler.PowerMateEventHandler \textit{(class)}!PowerMateEventHandler.PowerMateEventHandler.stop \textit{(method)}}

    \vspace{0.5ex}

\hspace{.8\funcindent}\begin{boxedminipage}{\funcwidth}

    \raggedright \textbf{stop}(\textit{self})

    \vspace{-1.5ex}

    \rule{\textwidth}{0.5\fboxrule}
\setlength{\parskip}{2ex}
    Stop the capture/queuing of events.

\setlength{\parskip}{1ex}
    \end{boxedminipage}

    \label{PowerMateEventHandler:PowerMateEventHandler:get_next}
    \index{PowerMateEventHandler \textit{(module)}!PowerMateEventHandler.PowerMateEventHandler \textit{(class)}!PowerMateEventHandler.PowerMateEventHandler.get\_next \textit{(method)}}

    \vspace{0.5ex}

\hspace{.8\funcindent}\begin{boxedminipage}{\funcwidth}

    \raggedright \textbf{get\_next}(\textit{self}, \textit{block}={\tt True}, \textit{timeout}={\tt None})

    \vspace{-1.5ex}

    \rule{\textwidth}{0.5\fboxrule}
\setlength{\parskip}{2ex}
    Pull the next consolidated event off the queue, and return it.

\setlength{\parskip}{1ex}
      \textbf{Parameters}
      \vspace{-1ex}

      \begin{quote}
        \begin{Ventry}{xxxxxxx}

          \item[block]

          block until next is available

            {\it (type=bool)}

          \item[timeout]

          block for this long

            {\it (type=double)}

        \end{Ventry}

      \end{quote}

      \textbf{Return Value}
    \vspace{-1ex}

      \begin{quote}
      If start was run with rawOnly=True, an evdev.events.InputEvent; 
      Otherwise, a ConsolidatedEventCode. In either case, None if there is 
      not an event ready and block is False, or timeout is reached.

      {\it (type=evdev.events.InputEvent, ConsolidatedEventCode, or None)}

      \end{quote}

\textbf{Note:} block and timeout are passed directly to queue.get(). If block is TRUE, the
thread will block for timeout seconds for the next event. If timeout is 
None, it will wait indefinitely. If block is False, an event will be 
grabbed only if one is ready immediately.



    \end{boxedminipage}

    \label{PowerMateEventHandler:PowerMateEventHandler:set_turn_delay}
    \index{PowerMateEventHandler \textit{(module)}!PowerMateEventHandler.PowerMateEventHandler \textit{(class)}!PowerMateEventHandler.PowerMateEventHandler.set\_turn\_delay \textit{(method)}}

    \vspace{0.5ex}

\hspace{.8\funcindent}\begin{boxedminipage}{\funcwidth}

    \raggedright \textbf{set\_turn\_delay}(\textit{self}, \textit{delay})

    \vspace{-1.5ex}

    \rule{\textwidth}{0.5\fboxrule}
\setlength{\parskip}{2ex}
    Set the delay between when consolidated events will be registered.

    In an effort to reduce spam from a failry sensative device, this 
    variable was created. If multiple turn events come in, the first will 
    register a consolidated event, and those that come in within the delay 
    time will be ignored. Once the delay threshold has been reached, 
    another consolidated event will be registered.

\setlength{\parskip}{1ex}
      \textbf{Parameters}
      \vspace{-1ex}

      \begin{quote}
        \begin{Ventry}{xxxxx}

          \item[delay]

          time in ms between turn events.

            {\it (type=double)}

        \end{Ventry}

      \end{quote}

    \end{boxedminipage}

    \label{PowerMateEventHandler:PowerMateEventHandler:set_read_delay}
    \index{PowerMateEventHandler \textit{(module)}!PowerMateEventHandler.PowerMateEventHandler \textit{(class)}!PowerMateEventHandler.PowerMateEventHandler.set\_read\_delay \textit{(method)}}

    \vspace{0.5ex}

\hspace{.8\funcindent}\begin{boxedminipage}{\funcwidth}

    \raggedright \textbf{set\_read\_delay}(\textit{self}, \textit{delay})

    \vspace{-1.5ex}

    \rule{\textwidth}{0.5\fboxrule}
\setlength{\parskip}{2ex}
    This was intendted to allow the reading of events to be stoppable (i.e 
    to keep from blocking the thread indefinitely). It was made tunable to 
    allow good performance on fast CPUs, but not hog resources on slower 
    machines.

    Setting delay to None will cause the thread to block indefinitely. This
    will probably yield the best performance, but means the thread will not
    stop after a call to stop() until a new event is triggered.

\setlength{\parskip}{1ex}
      \textbf{Parameters}
      \vspace{-1ex}

      \begin{quote}
        \begin{Ventry}{xxxxx}

          \item[delay]

          Time in seconds to wait for the device to be readable.

            {\it (type=double)}

        \end{Ventry}

      \end{quote}

    \end{boxedminipage}

    \label{PowerMateEventHandler:PowerMateEventHandler:set_double_click_time}
    \index{PowerMateEventHandler \textit{(module)}!PowerMateEventHandler.PowerMateEventHandler \textit{(class)}!PowerMateEventHandler.PowerMateEventHandler.set\_double\_click\_time \textit{(method)}}

    \vspace{0.5ex}

\hspace{.8\funcindent}\begin{boxedminipage}{\funcwidth}

    \raggedright \textbf{set\_double\_click\_time}(\textit{self}, \textit{time})

    \vspace{-1.5ex}

    \rule{\textwidth}{0.5\fboxrule}
\setlength{\parskip}{2ex}
\setlength{\parskip}{1ex}
      \textbf{Parameters}
      \vspace{-1ex}

      \begin{quote}
        \begin{Ventry}{xxxx}

          \item[time]

          (in s) the button must be pressed again after a single press to 
          register as a double

            {\it (type=double)}

        \end{Ventry}

      \end{quote}

    \end{boxedminipage}

    \label{PowerMateEventHandler:PowerMateEventHandler:set_long_click_time}
    \index{PowerMateEventHandler \textit{(module)}!PowerMateEventHandler.PowerMateEventHandler \textit{(class)}!PowerMateEventHandler.PowerMateEventHandler.set\_long\_click\_time \textit{(method)}}

    \vspace{0.5ex}

\hspace{.8\funcindent}\begin{boxedminipage}{\funcwidth}

    \raggedright \textbf{set\_long\_click\_time}(\textit{self}, \textit{time})

    \vspace{-1.5ex}

    \rule{\textwidth}{0.5\fboxrule}
\setlength{\parskip}{2ex}
\setlength{\parskip}{1ex}
      \textbf{Parameters}
      \vspace{-1ex}

      \begin{quote}
        \begin{Ventry}{xxxx}

          \item[time]

          (in s) the button must be held to register a long press

            {\it (type=double)}

        \end{Ventry}

      \end{quote}

    \end{boxedminipage}

    \index{PowerMateEventHandler \textit{(module)}!PowerMateEventHandler.PowerMateEventHandler \textit{(class)}|)}
    \index{PowerMateEventHandler \textit{(module)}|)}
